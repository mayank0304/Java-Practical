\documentclass[a4paper,12pt]{article}
\usepackage{graphicx}
\usepackage{float}
\usepackage{listings}
\usepackage{color}
\usepackage{courier}
\usepackage{geometry}
\geometry{margin=1in}
\usepackage{enumitem}
\usepackage{titlesec}

% Define Java syntax highlighting
\definecolor{javared}{rgb}{0.6,0,0}
\definecolor{javagreen}{rgb}{0.25,0.5,0.35}
\definecolor{javapurple}{rgb}{0.5,0,0.35}
\definecolor{javadocblue}{rgb}{0.25,0.35,0.75}
\lstset{
    language=Java,
    basicstyle=\ttfamily\small,
    keywordstyle=\color{javared}\bfseries,
    stringstyle=\color{javagreen},
    commentstyle=\color{javagreen}\itshape,
    morecomment=[s][\color{javadocblue}]{/**}{*/},
    numbers=left,
    numberstyle=\tiny\color{black},
    stepnumber=1,
    numbersep=10pt,
    tabsize=4,
    showspaces=false,
    showstringspaces=false
}

% Custom command for practical title
\newcounter{practicalno} % Create a new counter for practical numbers
\setcounter{practicalno}{-1} % Initialize the counter to 0
\newcommand{\practicaltitle}[1]{
    \stepcounter{practicalno} % Increment the practical number counter
    \newpage
    \begin{center}
        \vspace{1cm}
        \Large\textbf{Practical \thepracticalno} \\
        \vspace{0.5cm}
        \Large\textbf{#1} % Display the title on the next line
        \normalsize\vspace{1cm}
    \end{center}
}

%Redefine \subsection command to use hashtags instead of numbers
\titleformat{\subsection}[block]{\bfseries\large}{\texttt{\#}}{1em}{}
% Redefine \subsubsection command to use arrow instead of numbers
\titleformat{\subsubsection}[block]{\bfseries\normalsize}{\texttt{>}}{1em}{}

\title{\textbf{Java Practical}}
\author{Mayank}
\date{}

% Redefine \subsection command to use hashtags instead of numbers
\titleformat{\subsection}[block]{\bfseries\large}{\texttt{\#}}{1em}{}

\begin{document}

\maketitle

\practicaltitle{Introduction to Java}

\section{Introduction to Java}
JAVA was developed by James Gosling at Sun Microsystems Inc in May 1995 and later acquired by Oracle Corporation. It is a simple programming language. Java makes writing, compiling, and debugging programming easy. It helps to create reusable code and modular programs. Java is a class-based, object-oriented programming language and is designed to have as few implementation dependencies as possible. A general-purpose programming language made for developers to write once run anywhere that is compiled Java code can run on all platforms that support Java. Java applications are compiled to byte code that can run on any Java Virtual Machine. The syntax of Java is similar to C/C++.

Java is widely used for developing applications for desktop, web, and mobile devices. Java is known for its simplicity, robustness, and security features, making it a popular choice for enterprise-level applications.

\section{Java Syntax}
Java syntax is the set of rules defining how a Java program is written and interpreted.
\subsubsection{Code: }
\begin{lstlisting}
public class Syntax {
    public static void main(String[] args) {
        System.out.println("Hello, World!");
    }
}
\end{lstlisting}

% \subsection{Output: }
% \begin{}
%     \fbox{
%         \parbox{0.9\linewidth}{
%             \centering
%             \texttt{Hello, World!}
%         }
%     }
% \end{}

\subsection*{public class Main}
\begin{itemize}[leftmargin=2cm]
    \item \textbf{public}: An access modifier indicating that the class is accessible from other classes.
    \item \textbf{class}: A keyword used to define a class in Java.
    \item \textbf{Main}: The name of the class. By convention, class names in Java start with an uppercase letter.
\end{itemize}

\subsection*{public static void main(String[] args)}
\begin{itemize}[leftmargin=2cm]
    \item \textbf{static}: A keyword indicating that the method belongs to the class, not to instances of the class. It can be called without creating an object of the class.
    \item \textbf{void}:  The return type of the method, indicating that it does not return any value.
    \item \textbf{main}: The name of the method. This is the entry point of any Java application.
    \item \textbf{String[] args}: An array of String arguments passed to the method. These are command-line arguments.
\end{itemize}

\subsection*{System.out.println("Hello, World!")}
\begin{itemize}[leftmargin=2cm]
    \item \textbf{System}: A built-in class in the \texttt{java.lang} package.
    \item \textbf{out}: A static field in the \textbf{System} class, which is an instance of \textbf{PrintStream}.
    \item \textbf{println}: A method of \textbf{PrintStream} that prints a message to the standard output (usually the console) followed by a newline.
    \item \textbf{"Hello, World!"}: A string literal that is the message to be printed.
\end{itemize}


\section{Variables in Java}
Variables are containers for storing data values. In Java, every variable must be declared before it is used. A variable declaration includes the data type followed by the variable name. Java supports different types of variables, including:
\begin{itemize}[leftmargin=2cm]
    \item \textbf{Local Variables}: Declared inside a method and accessible only within that method.
    \item \textbf{Instance Variables}: Declared inside a class but outside any method. They are accessible from any method in the class.
    \item \textbf{Static Variables}: Declared with the \texttt{static} keyword. These are shared among all instances of the class.
\end{itemize}


\setcounter{section}{0}
\practicaltitle{Handling Various Data Types}

\section{Data Types in Java}
Java has two categories of data types: \textbf{Primitive Data Types} and \textbf{Reference/Object Data Types}.

\subsection{Primitive Data Types}
Primitive data types are the most basic data types available in Java.
\subsubsection{Code: }
\begin{lstlisting}
public class Data {
    public static void main(String[] args) {
        int myNum = 5;               
        float myFloatNum = 5.99f;
        double myDoubNum = 5.9999d;  
        char myLetter = 'D';        
        boolean myBool = true;       
        String myText = "Hello";     

        System.out.println(myNum);
        System.out.println(myFloatNum);
        System.out.println(myDoubNum);
        System.out.println(myLetter);
        System.out.println(myBool);
        System.out.println(myText);
    }
}
\end{lstlisting}

\begin{figure}[H]
    \centering
    \includegraphics[width=0.9\linewidth]{images/DataOutput.png}
    \caption{Primitve Data Types}
    \label{fig:sample_image}
\end{figure}

\begin{itemize}[leftmargin=2cm]
    \item \textbf{byte}: 8-bit signed integer. Range: -128 to 127.
    \item \textbf{short}: 16-bit signed integer. Range: -32,768 to 32,767.
    \item \textbf{int}: 32-bit signed integer. Range: -2\textsuperscript{31} to 2\textsuperscript{31}-1.
    \item \textbf{long}: 64-bit signed integer. Range: -2\textsuperscript{63} to 2\textsuperscript{63}-1.
    \item \textbf{float}: 32-bit floating-point number.
    \item \textbf{double}: 64-bit floating-point number.
    \item \textbf{char}: 16-bit Unicode character.
    \item \textbf{boolean}: Represents two values: true and false.
\end{itemize}

\subsection{Non Primitive Data Types}
Reference types in Java are Strings and arrays:
\subsubsection{Code: }
\begin{lstlisting}
public class NonPrimitive {
    public static void main(String[] args) {
        // String Data Type
        String stringVar = "Hello, Java!";
        System.out.println("\nString Data Type:");
        System.out.println("String: " + stringVar);

        // Array Data Type
        int[] intArray = {1, 2, 3, 4, 5};
        System.out.println("\nArray Data Type:");
        System.out.print("intArray: ");
        for (int num : intArray) {
            System.out.print(num + " ");
        }
        System.out.println();
    }
}  
\end{lstlisting}

\begin{figure}[H]
    \centering
    \includegraphics[width=0.9\linewidth]{images/NonPrimData.png}
    \caption{Non-Primitve Data Types}
    \label{fig:sample_image}
\end{figure}

\begin{itemize}[leftmargin=2cm]
    \item \textbf{Strings}: Sequences of characters.
    \item \textbf{Arrays}: Containers that hold multiple values of the same type.
\end{itemize}

\setcounter{section}{0}

% \practicaltitle{Practical 2: Type casting}
\practicaltitle{Type casting}
Type casting is when you assign a value of one primitive data type to another type. There are two types of type casting, implicit typecasting and explicit typecasting which are explained below:

\section{Implicit Type Casting}
Implicit type casting is done automatically when passing a smaller size type to a larger size type.
\begin{center}
    \fbox{
        \parbox{0.9\linewidth}{
            \centering
            \texttt{byte $\rightarrow$ short $\rightarrow$ char $\rightarrow$ int $\rightarrow$ long $\rightarrow$ float $\rightarrow$ double}
        }
    }
\end{center}
\subsubsection{Code: }
\begin{lstlisting}
public class Implicit {
    public static void main(String[] args) {
        int myInt = 9;
        double myDouble = myInt; // Automatic casting: int to double
        System.out.println(myInt); // Outputs 9
        System.out.println(myDouble); // Outputs 9.0
    }
}    
\end{lstlisting}
\subsubsection{Output: }
\begin{figure}[H]
    \centering
    \includegraphics[width=0.9\linewidth]{images/ImplicitOut.png}
    \caption{Implicit Type Conversion}
    \label{fig:sample_image}
\end{figure}

\section{Explicit Type Casting}
Explicit type casting must be done manually by placing the type in parentheses () in front of the value.
\begin{center}
    \fbox{
        \parbox{0.9\linewidth}{
            \centering
            \texttt{double $\rightarrow$ float $\rightarrow$ long $\rightarrow$ int $\rightarrow$ char $\rightarrow$ short $\rightarrow$ byte}
        }
    }
\end{center}
\begin{lstlisting}
public class Explicit {
    public static void main(String[] args) {
        double myDouble = 9.78d;
        int myInt = (int) myDouble; // Manual casting: double to int
    
        System.out.println(myDouble);   // Outputs 9.78
        System.out.println(myInt);      // Outputs 9
    }
}
\end{lstlisting}
\subsubsection{Output: }
\begin{figure}[H]
    \centering
    \includegraphics[width=0.9\linewidth]{images/ExplicitOut.png}
    \caption{Explicit Type Conversion}
    \label{fig:sample_image}
\end{figure}

\setcounter{section}{0}

\practicaltitle{Practical 3: Array 1D and 2D}

\section{1-Dimensional Array}
Arrays are used to store multiple values in a single variable, instead of declaring separate variables for each value.
\subsubsection{Code: }
\begin{lstlisting}
public class Array {
    public static void main(String[] args) {
        String[] cars = {"Volvo", "BMW", "Ford", "Mazda"};
        System.out.println(cars[0]); 
        System.out.println(cars[1]); 
        System.out.println(cars[2]); 
        System.out.println(cars[3]); 
        // Changing an element of an array
        cars[0] = "audi";
        System.out.println(cars[0]); 
        // Length of an array
        System.out.println(cars.length);
        // Loop through an array
        for (String arr  : cars) {
            System.out.println(arr);
        }
    }
}  
\end{lstlisting}
\subsubsection{Output: }
\begin{figure}[H]
    \centering
    \includegraphics[width=0.9\linewidth]{images/output2.png}
    \caption{output of 1-D array}
    \label{fig:sample_image}
\end{figure}

\section{Multi Dimensional Array}
A multidimensional array is an array of arrays. Multidimensional arrays are useful when you want to store data as a tabular form, like a table with rows and columns.
\subsubsection{Code: }
\begin{lstlisting}
    // Program for Multi-Dimensional Array

public class Array2D {
    public static void main(String[] args) {
        int[][] my2DArr = {{10, 20, 30, 40}, {50, 60, 70}};
        // Accessing Elemensts of array
        System.out.println(my2DArr[0][0]); // 10
        System.out.println(my2DArr[1][2]); // 70
        // change element of array
        my2DArr[0][0] = 100;
        System.out.println(my2DArr[0][0]); //100

        // Loop through a multi dimensional array
        System.out.println("Looping through an array");
        for (int[] row : my2DArr) {
            for(int i : row) {
                System.out.println(i);
            }
        }
    }
}
  
\end{lstlisting}
\subsubsection{Output: }
\begin{figure}[H]
    \centering
    \includegraphics[width=0.9\linewidth]{images/output3.png}
    \caption{output of Multi-D array}
    \label{fig:sample_image}
\end{figure}

\setcounter{section}{0}

\practicaltitle{Various Control Strucutures}

\section{For loop}
For loop provides a concise way of writing the loop structure. Unlike a while loop, a
for statement consumes the initialization, condition and increment/decrement in one line
thereby providing a shorter, easy to debug structure of looping.
\subsubsection{Code: }
\begin{lstlisting}
import java.util.Scanner;

public class ForLoop {

    public static void main(String[] args) {
        Scanner sc = new Scanner(System.in);
        System.out.print("Enter a number: ");
        int n = sc.nextInt();
        for (int i = 0; i < n; i++) {
            System.out.print(i + " ");
        }
        System.out.println();
    }
}
\end{lstlisting}
\subsubsection{Output: }
\begin{figure}[H]
    \centering
    \includegraphics[width=0.9\linewidth]{images/ForOut.png}
    \caption{output of for loop}
    \label{fig:sample_image}
\end{figure}

\section{While loop}
A while loop is a control flow statement that allows code to be executed repeatedly
based on a given Boolean condition. The while loop can be thought of as a repeating if
statement.
\subsubsection{Code: }
\begin{lstlisting}
import java.util.Scanner;

public class WhileLoop {

    public static void main(String[] args) {
// Sum of first n numbers
        Scanner sc = new Scanner(System.in);
        System.out.print("Enter a number: ");
        int n = sc.nextInt();
        int sum = 0;
        while (n > 0) {
            sum += n--;
        }
        System.out.println(sum);
    }
}
\end{lstlisting}
\subsubsection{Output: }
\begin{figure}[H]
    \centering
    \includegraphics[width=0.9\linewidth]{images/WhileOut.png}
    \caption{output of while loop}
    \label{fig:sample_image}
\end{figure}

\section{Do-While loop}
Do-While loop is similar to while loop with only difference that it checks for condition
after executing the statements, and therefore is an example of Exit Control Loop.
\subsubsection{Code: }
\begin{lstlisting}
import java.util.Scanner;

public class DoWhile {

    public static void main(String[] args) {
// Sum of first n numbers
        Scanner sc = new Scanner(System.in);
        System.out.print("Enter a number: ");
        int n = sc.nextInt();
        int sum = 0;
        do {
            sum += n--;
        } while (n > 0);
        System.out.println(sum);
    }
}
\end{lstlisting}
\subsubsection{Output: }
\begin{figure}[H]
    \centering
    \includegraphics[width=0.9\linewidth]{images/image.png}
    \caption{output of do-while loop}
    \label{fig:sample_image}
\end{figure}

\setcounter{section}{0}

\practicaltitle{Various Decision Strucutures}

\section{The IF statement}
Use the if statement to specify a block of Java code to be executed if a condition is true.
\subsubsection{Code: }
\begin{lstlisting}
public class If {
    public static void main(String[] args) {
        if (20 > 18) {
            System.out.println("20 is greater than 18");
            }
    }
}    
\end{lstlisting}
\subsubsection{Output: }
\begin{figure}[H]
    \centering
    \includegraphics[width=0.9\linewidth]{images/output4.png}
    \caption{output of if statement}
    \label{fig:sample_image}
\end{figure}

\section{The IF-Else statement}
Executes one block of code if its condition evaluates to true, and another block of code if it evaluates to false.
\subsubsection{Code: }
\begin{lstlisting}
public class IfElse {
    public static void main(String[] args) {
        int time = 20;
        if (time < 18) {
            System.out.println("Good day.");
        } else {
            System.out.println("Good evening.");
        }
    }
}    
\end{lstlisting}
\subsubsection{Output: }
\begin{figure}[H]
    \centering
    \includegraphics[width=0.9\linewidth]{images/output5.png}
    \caption{output of if-else statement}
    \label{fig:sample_image}
\end{figure}

\section{The IF-Else ladder}
Executes one block of code if its condition evaluates to true, and then checks other
coditions given in else if statements if it is false, or executes the last else block if nothing
is true
\subsubsection{Code: }
\begin{lstlisting}
import java.util.Scanner;

public class IfElseLad {

    public static void main(String[] args) {
        Scanner sc = new Scanner(System.in);
        System.out.print("Enter age -> ");
        int age = sc.nextInt();
        if (age < 12) {
            System.out.println("Child"); 
        }else if (age < 18) {
            System.out.println("Teenager"); 
        }else {
            System.out.println("Adult");
        }
    }
}

    \end{lstlisting}
\subsubsection{Output: }
\begin{figure}[H]
    \centering
    \includegraphics[width=0.9\linewidth]{images/IfElseLad.png}
    \caption{output of if-else-ladder statement}
    \label{fig:sample_image}
\end{figure}

\section{Nested If-Else}
We can put If-Else statements inside otehr If-Else statemetns in order to build more
complex logic
\subsubsection{Code: }
\begin{lstlisting}
import java.util.Scanner;

public class NestedIf {

    public static void main(String[] args) {
        Scanner sc = new Scanner(System.in);
        System.out.print("Enter a -> ");
        int a = sc.nextInt();
        System.out.print("Enter b -> ");
        int b = sc.nextInt();
        System.out.print("Enter c -> ");
        int c = sc.nextInt();
        if (a > b) {
            if (a > c) {
                System.out.println("A is greatest"); 
            }else {
                System.out.println("C is greatest");
            }
        } else {
            if (b > c) {
                System.out.println("B is greatest"); 
            }else {
                System.out.println("C is greatest");
            }
        }
    }
}    
\end{lstlisting}
\subsubsection{Output: }
\begin{figure}[H]
    \centering
    \includegraphics[width=0.9\linewidth]{images/NestOut.png}
    \caption{output of nested-if statement}
    \label{fig:sample_image}
\end{figure}

\section{Switch statement}
The switch statement in Java is a multi-way branch statement. In simple words, the Java
switch statement executes one statement from multiple conditions.
\subsubsection{Code: }
\begin{lstlisting}
import java.util.Scanner;

public class Switch {

    public static void main(String[] args) {
        Scanner sc = new Scanner(System.in);
        System.out.print("Enter a number: ");
        int day = sc.nextInt();
        switch (day) {
            case 1:
                System.out.println("Sunday");
                break;
            case 2:
                System.out.println("Monday");
                break;
            case 3:
                System.out.println("Tuesday");
                break;
            case 4:
                System.out.println("Wednesday");
                break;
            case 5:
                System.out.println("Thursday");
                break;
            case 6:
                System.out.println("Friday");
                break;
            case 7:
                System.out.println("Saturday");
                break;
            default:
                System.out.println("Invalid day");
        }
    }
}
    
\end{lstlisting}
\subsubsection{Output: }
\begin{figure}[H]
    \centering
    \includegraphics[width=0.9\linewidth]{images/SwitchOut.png}
    \caption{output of switch statement}
    \label{fig:sample_image}
\end{figure}

\end{document}